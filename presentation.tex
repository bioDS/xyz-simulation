\documentclass[8pt]{beamer}
\usetheme{default}

%\geometry{paperwidth=140mm,paperheight=105mm}

\title{Inferring genetic interactions using xyz}
\author{Kieran Elmes}
\begin{document}
\begin{frame}[plain]
    \maketitle
\end{frame}
\begin{frame}{What are we doing}
content...
\end{frame}
\begin{frame}{Why do we care}
content...
\end{frame}
\begin{frame}{$P = 100$}
Results for $p=100$ using vanilla xyz
\begin{center}
\begin{minipage}{0.9\linewidth}
	\centering
	\includegraphics[width=0.5\linewidth]{"PrecRecF1/PrecRecF1_n1000_tno-vanilla_xyz"}%
	\includegraphics[width=0.5\linewidth]{"PrecRecF1/PrecRecF1_n1000_tyes-vanilla_xyz"}
\end{minipage}
\end{center}
For 1000 measurements of only 100 genes, results are good.
\end{frame}
\begin{frame}{$P = 1000$}
Results for $p=1000$ using vanilla xyz
\begin{center}
	\begin{minipage}{0.9\linewidth}
		\centering
		\includegraphics[width=0.5\linewidth]{"PrecRecF1/PrecRecF1_n10000_tno-vanilla_xyz"}%
		\includegraphics[width=0.5\linewidth]{"PrecRecF1/PrecRecF1_n10000_tyes-vanilla_xyz"}
	\end{minipage}
\end{center}
Large data sets are a problem, however.
\end{frame}
\begin{frame}{$P = 1000$, increased limits in xyz}
Results for $p=1000$ using increased limits\footnote{commit 4517afbc59bdf091eed75683a28070476933f5da}
\begin{center}
	\begin{minipage}{0.9\linewidth}
		\centering
		\includegraphics[width=0.5\linewidth]{"PrecRecF1/increased_limits/PrecRecF1_n10000_tno"}%
		\includegraphics[width=0.5\linewidth]{"PrecRecF1/increased_limits/PrecRecF1_n10000_tyes"}
	\end{minipage}
\end{center}
Increasing the maximum allowed number of interactions in xyz does not improve results.
\end{frame}
\begin{frame}{Larger values of $L$}
\begin{center}
	\begin{minipage}{0.9\linewidth}
		\centering
		\includegraphics[width=0.5\linewidth]{"l_diff/l_diff_n10000_tno"}%
		\includegraphics[width=0.5\linewidth]{"l_diff/l_diff_n10000_tyes"}
	\end{minipage}
\end{center}


Where results are found, a larger value of $L$ does not obviously improve accuracy. It does seem to improve the odds of finding at least some results, however.
\end{frame}
\begin{frame}{Interaction strength}
\begin{center}
	\begin{minipage}{0.9\linewidth}
		\centering
		\includegraphics[width=0.5\linewidth]{"FXstrength/FXstrength_PRF_n10000_L100_tno"}%
		\includegraphics[width=0.5\linewidth]{"FXstrength/FXstrength_PRF_n10000_L100_tyes"}
	\end{minipage}
\end{center}


Fixing $L = 100$ (for the moment), a small number of strong interactions are somewhat reliably found. When the data contains a large enough number of total interactions, performance relatively poor.
\end{frame}

\end{document}

% Things to mention: (also see notes)
